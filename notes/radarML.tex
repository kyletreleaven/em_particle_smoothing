\documentclass[a4paper,10pt]{article}

\usepackage{amsmath}
\usepackage{amsfonts}
\usepackage{amssymb}
\usepackage{amsthm}
%\usepackage{subfigure}
%\usepackage{upgreek}
\usepackage{graphicx}

%\usepackage[dvips,pdftex]{hyperref}

\date{2011-01-26}

\begin{document}

\newcommand{\onehalf}{\frac{1}{2}}
%
\newcommand{\obs}{\text{o}}
\newcommand{\noise}{v}
\newcommand{\noisevar}{\sigma^2}
\newcommand{\normpdf}{{\mathcal N}}

\newcommand{\reals}{{\mathbb R}}

% physical notations
\newcommand{\pos}{{\bf x}}
\newcommand{\xvar}{x}
\newcommand{\yvar}{y}
\newcommand{\zvar}{z}
\newcommand{\slrange}{R}
\newcommand{\altitude}{\zvar}
\newcommand{\bearing}{\theta}

\newcommand{\likelihood}{L}

\newcommand{\argmin}{\operatorname*{arg\,min}}
\newcommand{\argmax}{\operatorname*{arg\,max}}




\section{Maximum Likelihood Projection of Radar Measurements}


A perfect radar gives the following measurements for an object at point (in $\reals^3$)
\[ \pos = (\xvar,\yvar,\zvar): \]
%
The \emph{slant range} is the straight-line distance from the origin to the object;
\[
  \slrange := \sqrt{ \xvar^2 + \yvar^2 + \zvar^2 } ;
\]
the \emph{altitude} is the vertical height
\[ \zvar := \zvar; \]
the \emph{bearing} is the angle
\[ \bearing := \arctan2(\yvar,\xvar). \]

\newcommand{\obstuple}{{(\slrange^\obs,\altitude^\obs,\bearing^\obs)}}
\newcommand{\noisetuple}{{(\noise_\slrange,\noise_\altitude,\noise_\bearing)}}

Consider a noisy radar which gives readings $\obstuple$,
where
\[
\begin{aligned}
  \slrange^\obs = \slrange + \noise_\slrange	\\
  \altitude^\obs = \altitude + \noise_\altitude	\\
  \bearing^\obs = \bearing + \noise_\bearing;
\end{aligned}
\]
%
Here, $\noisetuple$ denotes the sensor \emph{noise}, with
\[
  \noise_\slrange \sim \normpdf(0,\noisevar_\slrange),
  \quad
  \noise_\altitude \sim \normpdf(0,\noisevar_\altitude),
  \quad
  \noise_\bearing \sim \normpdf(0,\noisevar_\bearing),
\]
and so given position $\pos$, the distribution of $\obstuple$
has probability density function (pdf)
\[
  \likelihood(\slrange^\obs,\altitude^\obs,\bearing^\obs ; \pos ) :=
  \normpdf( \slrange^\obs ; \slrange(\pos), \noisevar_\slrange )
  \times
  \normpdf( \altitude^\obs ; \altitude(\pos), \noisevar_\altitude )
  \times
  \normpdf( \bearing^\obs ; \bearing(\pos), \noisevar_\bearing ).
\]

We would like to obtain an \emph{estimate} $\hat\pos \equiv (\hat\xvar,\hat\yvar,\hat\zvar)$ of the position,
by the maximum likelihood (ML) projection of $\obstuple$
\[
\argmax_\pos \likelihood(\slrange^\obs,\altitude^\obs,\bearing^\obs ; \pos ).
\]
%
\newcommand{\xyrad}{r}
Substituting $\xyrad := \sqrt{ \xvar^2 + \yvar^2 }$,
we can write
\[
  \likelihood(\slrange^\obs,\altitude^\obs,\bearing^\obs ; \pos ) :=
  \normpdf( \slrange^\obs ; \slrange(\xyrad,\zvar), \noisevar_\slrange )
  \times
  \normpdf( \altitude^\obs ; \altitude(\xyrad,\zvar), \noisevar_\altitude )
  \times
  \normpdf( \bearing^\obs ; \bearing(\bearing), \noisevar_\bearing ).
\]
Clearly,
\[
  \hat\bearing := \bearing^\obs,
\]
and we can focus on the simpler problem
\[
  \argmax_{\xyrad,\zvar} \tilde\likelihood(\slrange^\obs,\altitude^\obs ; \xyrad, \zvar )
\]
where
\[
  \tilde\likelihood(\slrange^\obs,\altitude^\obs ; \xyrad, \zvar ) 
  :=
  \normpdf( \slrange^\obs ; \slrange(\xyrad,\zvar), \noisevar_\slrange )
  \times
  \normpdf( \altitude^\obs ; \zvar, \noisevar_\altitude ).
\]
%
This can be achieved by
\[
  \argmin_{\zvar', \slrange' \geq |\zvar'|}
  \frac{ \left( \slrange' - \slrange^\obs  \right)^2 }{ \noisevar_\slrange }
  +
  \frac{ \left(  \zvar' - \altitude^\obs \right)^2 }{ \noisevar_\altitude }.
\]

If $\slrange^\obs \geq |\altitude^\obs|$, then clearly $\hat\altitude = \altitude^\obs$
and $\hat\slrange = \slrange^\obs$.
%
If not, then the inequality constraint holds with equality,
and we have
\[
  \argmin_{\zvar'}
  \frac{ \left( |\zvar'| - \slrange^\obs \right)^2 }{ \noisevar_\slrange }
  +
  \frac{ \left(  \zvar'  - \altitude^\obs \right)^2 }{ \noisevar_\altitude }.
\]
%
We can split this into a two sided optimization,
taking the \emph{best} solution among
\[
  \argmin_{\zvar' \geq 0 }
  \frac{ \left( \zvar' - \slrange^\obs \right)^2 }{ \noisevar_\slrange }
  +
  \frac{ \left(  \zvar'  - \altitude^\obs \right)^2 }{ \noisevar_\altitude }.
\]
%
and
%
\[
  \argmin_{\zvar' \leq 0 }
  \frac{ \left( \zvar' + \slrange^\obs \right)^2 }{ \noisevar_\slrange }
  +
  \frac{ \left(  \zvar'  - \altitude^\obs \right)^2 }{ \noisevar_\altitude }.
\]








\end{document}
